\documentclass[paper=a4wide, fontsize=12pt]{scrartcl}	 % A4 paper and 11pt font size
\usepackage[svgnames]{xcolor} % Using colors
%%\usepackage{background} % To include background images
\usepackage{fancyhdr} % Needed to define custom headers/footers
\usepackage[a4paper, left=20mm, right=20mm, top=20mm, bottom=2.5cm, footskip=1.2cm]{geometry}  % Changing size of document


\usepackage{braket}


\pagestyle{fancy} % Enables the custom headers/footers
\lfoot{\color{Grey} Zhang Ying Huang}  % Write your name here
\rfoot{\color{Grey} Università degli studi dell'Insubria}
\cfoot{\color{Grey} \thepage}
\renewcommand{\headrulewidth}{0.0pt} % No header rule
\renewcommand{\footrulewidth}{0.4pt} % Thin footer rule

\title{ \color{DarkRed}Thesis work: comparing three algorithms for the median estimation on a random stream of data\vspace{-1.5em}}
\date{} % No date


%%%%%% Starting the document

\begin{document}

\maketitle % Print the title
\thispagestyle{fancy} % Enabling the custom headers/footers for the first page


% In the following lines, add the relevant information
\vspace{-0.5cm} \textbf{Student Name: Zhang Ying (Matteo) Huang}

\textbf{Student ID: 756483}

\textbf{Version: 1.0}

\vspace{0.5cm}

\section*{The idea behind the work}
What we want to investigate during this work is how three algorithms compare in the median estimation process on a stream of data.
The three algorithms considered are the following:
\begin{itemize}
    \item numpy.median();
    \item two-heaps median;
    \item $\mathbf{\varepsilon}$-median.
\end{itemize}
The parameters taken into consideration for comparing these algorithms are time-complexity and space-complexity.
A description for each algorithm follows.
%%to be revised
%% END OF SECTION


\section*{Numpy.median()}
%%%% TO DO - STUDY THE IMPLEMENTATION BEHIND THE ALGORITHM
%% END OF SECTION


\section*{A quick introduction to the heap data structure}
Before talking about the two-heaps, let's introduce the heap quickly.
So, what is a heap? \\ 
A heap is a tree-based data structure, which is binary and balanced. \\
The main operations we can perform are:
\begin{itemize}
    \item insertion: inserts the given element as a leaf and is then ordered accordingly;
    \item deletion: deletes the root element from the data structure, by swapping its value with the rightmost element, then an ordering process follows;
    \item peeking: returns the root value.
\end{itemize}

The first two operations rely on the following sub-operations (for simplicity, we suppose that each internal node, root included, has elements with a lesser value as children):
\begin{itemize}
    \item swim: after an insertion, if the order condition is broken, the inserted element climbs up the tree until it's positioned correctly according to the heap order property, which will be discussed later;
    \item sink: after the deletion, since the smallest element is at the top, the order condition is broken, therefore it has to go down the nodes until the right position is reached.
\end{itemize}
%%to be revised
%% END OF SECTION


\section*{Heap properties}
There are two types of heaps:
\begin{itemize}
    \item the max-heap: each internal node, root included, has smaller children;
    \item the min-heap: each internal node, root included, has bigger children.
\end{itemize}

The heap is usually represented as an array and, supposing that the first index is 1, each element at position i has, as children, the elements at position 2i and 2i+1.\\
The children don't need to follow any particular order criteria, as long as they're both bigger or smaller than the parent node. \\

As for the computational complexity, space-wise, it's $\mathcal{O}(N)$, while, time-wise, it's $\mathcal{O}(log N)$ for both insertion and deletion, and $\mathcal{O}(1)$ for peeking.\\
In this context, the deletion is not needed, as we suppose the data flow is continuous and the two-heap data structure grows accordingly.
%%to be revised
%% END OF SECTION


\section*{Two-heaps median}
Assuming that one knows the heap data-structure, this algorithm, as the name suggests, is based on two heaps, specifically on a max-heap and on a min-heap. \\
Taking as example a finite array A = \{1, 2, 3, 4, 5\} (the same concept applies to a stream of data), the max-heap will contain the lower-half of the array, while the min-heap will contain the upper-half. \\
Right after each insertion, the balance condition is checked, for which either of the following condition must be true:
\begin{itemize}
    \item the max heap can have, at most, one more element than the min-heap;
    \item both heaps have the same number of elements.
\end{itemize}

If none of the previously stated condition is met, a balancing operation is carried out.
Let's consider the following cases:
\begin{itemize}
    \item the max-heap has two more elements than the min-heap: the root of the max-heap is popped and inserted into the min-heap; 
    \item the min-heap has one more element than the max-heap: the root of the min-heap is popped and inserted into the max-heap.
\end{itemize} 

The composite operation pop+insertion is called "poll".

Taking the previously-declared array A, we will have the following situation (the following arrays are heap-ordered):
\begin{itemize}
    \item max-heap = \{3, 1, 2\}
    \item min-heap = \{4, 5\} 
\end{itemize}
At this point, if we want to estimate the median, the algorithm just takes the root of the max-heap and return it as the output (3, in our case). \\
Suppose we want to insert another number, say 6 for simplicity. What happens is the element is compared to the root of the max-heap and, if the value of the element we want to insert is less or equal than the considered root, we insert it into the max-heap, otherwise, we insert it into the min-heap. \\
So, 6 is first compared to the max-heap root and, since it's a greater value, it's directly inserted into the min-heap: 
\begin{itemize}
    \item max-heap = \{3, 1, 2\}
    \item min-heap = \{4, 5, 6\} 
\end{itemize}

Now we have an even number of elements spread across the two heaps. Say we want to estimate the median and what happens is that the root of both heaps are taken, then the mean value of the two elements is computed and returned as output (in our case, (4+3)/2 = 3.5).
\\
Now, let's discuss complexity:\\
Being N the number of elements, the space-complexity is, obviously, $\mathcal{O}(N)$ 
While, talking about time-complexity, we have to consider each operation separately first:
\begin{itemize}
    \item insertion of an element: given that the underlying data-structure is a heap, we have a time complexity of $\mathcal{O}(\log N)$ for a single insertion. Since we have two heaps, there's a multiplicative constant of 2, taking the complexity up to $\mathcal{O}(2\log N)$.\\
    Extending this operation to N elements, we will have a final time-complexity of $\mathcal{O}(2N\log N)$;
    \item balancing: considering this operation relies on the heap-push operation and that the two heaps are involved most of the times, the time-complexity is $\mathcal{O}(2\log N)$;
    \item polling: this operation takes $\mathcal{O}(\log N)$ in time, since it relies on the heap-pop operation. Specifically, it takes $\mathcal{O}(1)$ for retrieving the root and $\mathcal{O}(\log N)$ for re-balancing the heap;
    \item peeking: this only takes $\mathcal{O}(1)$.
\end{itemize}
Considering the case of a random stream of data, we can expect the algorithm to run with the following time-complexity: N * ($2\log N + 2\log N + \log N$) = $\Theta(5N\log N)$.
%% END OF SECTION
%%to be revised


\section*{The importance of the balancing operation in the two-heaps}
%% also talk about the importance of not deleting the elements at a certain delta-t
%% add demostration with images
%% END OF SECTION


\section*{$\mathbf{\varepsilon}$-median}
%% to do
%% END OF SECTION


\section*{Comparing the three algorithms}
%% to do
%% END OF SECTION


\section*{Investigating the behavior of $\mathbf{\varepsilon}$ by varying the constant k}
We set the $\mathbf{\varepsilon}$ value to be dependent from two parameters: the first generated value $x_0$ and an arbitrary large constant k.
In order to see which value to take for $\mathbf{\varepsilon}$, we investigate how it's affected from the choice of k, by taking 10 different values (250, 500, 750, $10^3$, $2*10^3$, $5*10^3$, $10^4$, $2*10^4$, $5*10^4$, $10^5$). %%to be continued
%% END OF SECTION


\section*{Percentage deviation plots}
Here we want to consider N = 1000 median-estimation trials, fix an iteration number i (let's say 800), and take the i-th value $x_i$ in every trial; all the other data are thrown away.
Then we want to take the residuals by applying the formula $\frac{x_i}{\mu} - 1$ (we scale the values to a percentage format). 
Each trial has a generation step of 1000 data points, so we end up having $10^6$ operations.\\
This process is applied to all three algorithms, and, in the $\mathbf{\varepsilon}$-median case, we consider all the k values that influence the computation of the parameter $\mathbf{\varepsilon}$.
%%to be continued
%% END OF SECTION
\end{document}
